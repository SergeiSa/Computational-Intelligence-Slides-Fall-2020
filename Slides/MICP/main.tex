\documentclass{beamer}

 \pdfmapfile{+sansmathaccent.map}


\mode<presentation>
{
  \usetheme{Warsaw} % or try Darmstadt, Madrid, Warsaw, Rochester, CambridgeUS, ...
  \usecolortheme{crane} % or try seahorse, beaver, crane, wolverine, ...
  \usefonttheme{serif}  % or try serif, structurebold, ...
  \setbeamertemplate{navigation symbols}{}
  \setbeamertemplate{caption}[numbered]
} 

\renewcommand{\familydefault}{\rmdefault}


\hypersetup{
    colorlinks=true,
    linkcolor=blue,
    filecolor=blue,      
    urlcolor=blue,
}

\usepackage{amsmath}
\usepackage{mathtools}

\DeclareMathOperator*{\argmin}{arg\,min}

\usepackage{subcaption}

\usepackage{tikz}
\tikzset{every picture/.style={line width=0.75pt}} %set default line width to 0.75pt        
%%%%%%%%%%%%%%%%%%%%%%%%%%%%%%%%%%%%%%%%%%%%%%%%%%%


\usepackage{listings}
\usepackage{color}
\definecolor{mygreen}{rgb}{0,0.6,0}
\definecolor{mygray}{rgb}{0.5,0.5,0.5}
\definecolor{mymauve}{rgb}{0.58,0,0.82}

\definecolor{cof}{RGB}{255,187,0}
\definecolor{pur}{RGB}{205,137,0}
\definecolor{greeo}{RGB}{91,173,69}
\definecolor{greet}{RGB}{52,111,72}

\usetikzlibrary{fadings}
\usetikzlibrary{patterns}
\usetikzlibrary{shadows.blur}
\usetikzlibrary{shapes}

\lstset{ 
  backgroundcolor=\color{white},   % choose the background color; you must add \usepackage{color} or \usepackage{xcolor}; should come as last argument
  basicstyle=\footnotesize,        % the size of the fonts that are used for the code
  breakatwhitespace=false,         % sets if automatic breaks should only happen at whitespace
  breaklines=true,                 % sets automatic line breaking
  captionpos=b,                    % sets the caption-position to bottom
  commentstyle=\color{mygreen},    % comment style
  deletekeywords={...},            % if you want to delete keywords from the given language
  escapeinside={\%*}{*)},          % if you want to add LaTeX within your code
  extendedchars=true,              % lets you use non-ASCII characters; for 8-bits encodings only, does not work with UTF-8
  firstnumber=0000,                % start line enumeration with line 0000
  frame=single,	                   % adds a frame around the code
  keepspaces=true,                 % keeps spaces in text, useful for keeping indentation of code (possibly needs columns=flexible)
  keywordstyle=\color{blue},       % keyword style
  language=Octave,                 % the language of the code
  morekeywords={*,...},            % if you want to add more keywords to the set
  numbers=left,                    % where to put the line-numbers; possible values are (none, left, right)
  numbersep=5pt,                   % how far the line-numbers are from the code
  numberstyle=\tiny\color{mygray}, % the style that is used for the line-numbers
  rulecolor=\color{black},         % if not set, the frame-color may be changed on line-breaks within not-black text (e.g. comments (green here))
  showspaces=false,                % show spaces everywhere adding particular underscores; it overrides 'showstringspaces'
  showstringspaces=false,          % underline spaces within strings only
  showtabs=false,                  % show tabs within strings adding particular underscores
  stepnumber=2,                    % the step between two line-numbers. If it's 1, each line will be numbered
  stringstyle=\color{mymauve},     % string literal style
  tabsize=2,	                   % sets default tabsize to 2 spaces
  title=\lstname                   % show the filename of files included with \lstinputlisting; also try caption instead of title
}




%%%%%%%%%%%%%%%%%%%%%%%%%%%%%%%%%%%%%%%%%%%%%%%%%%%%%


\title{ Mixed Integer Convex Programming }
\subtitle{Computational Intelligence, Lecture 11}
\author{by Sergei Savin}
\centering
\date{Fall 2020}



\begin{document}
\maketitle


\begin{frame}{Content}

\begin{itemize}
\item Mixed Integer Linear Programming (MILP)
\item Mixed Integer Quadratic Programming (MIQP)
\item Remarks
\item Example: Footstep planning
\item Big-M method relaxation
\begin{itemize}
    \item Basic idea
    \item Systems of inequalities
    \item Illustration, part 1
    \item Illustration, part 2
    \item Multiple variables
\end{itemize}
\item  Example: Footstep planning
\begin{itemize}
    \item Formulation as MIQP
    \item Evenly spaced steps
    \item Code, part 1
    \item Code, part 2
    \item Code, part 3
\end{itemize}
\item Homework
\end{itemize}

\end{frame}



\begin{frame}{Mixed Integer Linear Programming (MILP)}
\framesubtitle{General form}
\begin{flushleft}

A general form of a mixed-integer linear program is:

%
\begin{equation} \label{LP}
\begin{aligned}
& \underset{\mathbf{x}, \mathbf{y}}{\text{minimize}}
& & \mathbf{f}_1^\top \mathbf{x} + \mathbf{f}_2^\top \mathbf{y}, \\
& \text{subject to}
& & \begin{cases} 
\begin{bmatrix}
\mathbf{A}_{11} & \mathbf{A}_{12} \\
\mathbf{A}_{21} & \mathbf{A}_{22}
\end{bmatrix} 
\begin{bmatrix}
\mathbf{x} \\
\mathbf{y}
\end{bmatrix}
\leq 
\begin{bmatrix}
\mathbf{b}_1 \\
\mathbf{b}_2
\end{bmatrix}, \\ 
\\
\begin{bmatrix}
\mathbf{C}_{11} & \mathbf{C}_{12} \\
\mathbf{C}_{21} & \mathbf{C}_{22}
\end{bmatrix} 
\begin{bmatrix}
\mathbf{x} \\
\mathbf{y}
\end{bmatrix} = 
\begin{bmatrix}
\mathbf{d}_1 \\
\mathbf{d}_2
\end{bmatrix},  \\
\mathbf{y} \in \mathbb{N}^n.
\end{cases}
%
\end{aligned}
\end{equation}
 
In other words, the only difference is that some of the variables (denoted above as $\mathbf{y}$) are only allowed to assume pure integer values.
 
\end{flushleft}
\end{frame}



\begin{frame}{Mixed Integer Quadratic Programming (MIQP)}
\framesubtitle{General form}
\begin{flushleft}

A general form of a mixed-integer quadratic program is:

%
\begin{equation} \label{QP}
\begin{aligned}
& \underset{\mathbf{x}, \mathbf{y}}{\text{minimize}}
& & 
\begin{bmatrix}
\mathbf{x} & \mathbf{y}
\end{bmatrix}
\begin{bmatrix}
\mathbf{H}_{11} & \mathbf{H}_{12} \\
\mathbf{H}_{21} & \mathbf{H}_{22}
\end{bmatrix} 
\begin{bmatrix}
\mathbf{x} \\
\mathbf{y}
\end{bmatrix} +
\begin{bmatrix}
\mathbf{f}_1 & \mathbf{f}_2
\end{bmatrix}
\begin{bmatrix}
\mathbf{x} \\
\mathbf{y}
\end{bmatrix}, \\
& \text{subject to}
& & \begin{cases} 
\begin{bmatrix}
\mathbf{A}_{11} & \mathbf{A}_{12} \\
\mathbf{A}_{21} & \mathbf{A}_{22}
\end{bmatrix} 
\begin{bmatrix}
\mathbf{x} \\
\mathbf{y}
\end{bmatrix}
\leq 
\begin{bmatrix}
\mathbf{b}_1 \\
\mathbf{b}_2
\end{bmatrix}, \\ 
\\
\begin{bmatrix}
\mathbf{C}_{11} & \mathbf{C}_{12} \\
\mathbf{C}_{21} & \mathbf{C}_{22}
\end{bmatrix} 
\begin{bmatrix}
\mathbf{x} \\
\mathbf{y}
\end{bmatrix} = 
\begin{bmatrix}
\mathbf{d}_1 \\
\mathbf{d}_2
\end{bmatrix},  \\
\mathbf{y} \in \mathbb{N}^n.
\end{cases}
%
\end{aligned}
\end{equation}
 
Other mixed-integer convex programs can be constructed likewise  from their pure convex counterparts.
 
\end{flushleft}
\end{frame}



\begin{frame}{Remarks}
% \framesubtitle{General form}
\begin{flushleft}

\begin{itemize}
    \item Mixed-integer convex programs are not convex, even if the name seems to suggest otherwise. 
    \item The "convex program" in the phrase "mixed-integer convex program" should be understood as "if we fix values of the integer variables, or relax them into reals, the result will be a convex program".
    \item Mixed integer programs are usually solved using \emph{branch and bound algorithms}; in worse case scenario, the algorithms performs exhaustive search.
    \item In robotics applications, integer variables in mixed integer programs are often restricted to binary values, i.e. $\mathbf{y} \in \{0, 1\}^n.$. It is still called "mixed-integer" in the literature, although phrases such as "mixed-binary" or "binary constraints" are not rare.
\end{itemize}
 
\end{flushleft}
\end{frame}



\begin{frame}{Example: Footstep planning}
\framesubtitle{Problem statement}
\begin{flushleft}

Given $N$ convex regions defined by linear inequalities $\{ \mathbf{x}: \  \mathbf{A}_i \mathbf{x} \leq \mathbf{b}_i \}$ (which is called H-polytope representation), find a sequence of $K$ points (footsteps) from the given starting point to the given goal point, such that all footsteps lie in one of the convex regions.

\begin{equation}
\begin{aligned}
& \underset{\mathbf{x}_1, \ ..., \ \mathbf{x}_K}{\text{minimize}}
& & ||\mathbf{x}_1 - \mathbf{x}_{\text{start}}|| + ||\mathbf{x}_K - \mathbf{x}_{\text{goal}}||, \\
& \text{subject to}
& & \exists \{ \mathbf{A}_j, \mathbf{b}_j\} \in \Omega \ \text{s.t.} \ \mathbf{A}_j\mathbf{x}_i \leq \mathbf{b}_j
\end{aligned}
\end{equation}

\bigskip

where $\Omega = \{ \{ \mathbf{A}_1, \mathbf{b}_1\}, \ \{ \mathbf{A}_2, \mathbf{b}_2\}, \ ..., \ \{ \mathbf{A}_N, \mathbf{b}_N\} \}$. This is not a convex program.
 
\end{flushleft}
\end{frame}




\begin{frame}{Big-M method relaxation}
\framesubtitle{Basic idea}
\begin{flushleft}


One of the key methods associated with the use of mixed-integer programming in robotics is the big-M method. 

\bigskip

Assume you have two inequalities, $\mathbf{a}_1^\top \mathbf{x} \leq b_1$ and $\mathbf{a}_2^\top \mathbf{x} \leq b_2$, and you are happy if at least one of them holds. Define a new binary variables $c_1, c_2 \in \{0, \ 1 \}$ and find a big enough constant $M$, such that $\mathbf{a}_1^\top \mathbf{x} \leq b_1 + M$ and $\mathbf{a}_2^\top \mathbf{x} \leq b_2 + M$ holds for all of your domain (of the part of it you are interested in). Then you write your constraints as:

\begin{equation}
    \begin{cases}
    \mathbf{a}_1^\top \mathbf{x} \leq b_1 + M \cdot c_1 \\
    \mathbf{a}_2^\top \mathbf{x} \leq b_2 + M \cdot c_2 \\
    c_1 + c_2 = 1  \\
    c_i  \in \{0, \ 1 \}
    \end{cases}
\end{equation}

\end{flushleft}
\end{frame}




\begin{frame}{Big-M method relaxation}
\framesubtitle{Systems of inequalities}
\begin{flushleft}

It works the same way for the case when you have three (or two or more) systems of inequalities $\mathbf{A}_1 \mathbf{x} \leq \mathbf{b}_1$, $\mathbf{A}_2 \mathbf{x} \leq \mathbf{b}_2$, $\mathbf{A}_3 \mathbf{x} \leq \mathbf{b}_3$ and are happy if at least one holds:

\begin{equation}
    \begin{cases}
    \mathbf{A}_1 \mathbf{x} \leq \mathbf{b}_1 + M \cdot \mathbf{1} \cdot (1 - c_1) \\
    \mathbf{A}_2 \mathbf{x} \leq \mathbf{b}_2 + M \cdot \mathbf{1} \cdot (1 - c_2) \\
    \mathbf{A}_3 \mathbf{x} \leq \mathbf{b}_3 + M \cdot \mathbf{1} \cdot (1 - c_3) \\
    c_1 + c_2 + c_3 = 1 \\
    c_i  \in \{0, \ 1 \}
    \end{cases}
\end{equation}

where $\mathbf{1}$ is a vector of all ones. Notice that constraint $c_1 + c_2 + c_3 = 1$ can be replaced with $c_1 + c_2 + c_3 >= 1$, allowing avoid relaxing more than one region.

\end{flushleft}
\end{frame}



\begin{frame}{Big-M method relaxation}
\framesubtitle{Illustration, part 1}
\begin{flushleft}

Below are two H-polytops (convex regions represented by systems of inequalities):

\begin{figure} [h!]
\begin{center}
\includegraphics[width=2.6in]{fig1.png}
\end{center} 
% \caption{AR 601 bipedal robot, Innopolis University}
\end{figure}

As we can see, their union represents a non-convex domain, and they have no intersection.

\end{flushleft}
\end{frame}


\begin{frame}{Big-M method relaxation}
\framesubtitle{Illustration, part 2}
\begin{flushleft}

Now, one of them is relaxed as described above. Notice that the intersection of the two polytopes is non-relaxed polytope. 

\begin{figure}
\centering
\begin{subfigure}{.5\textwidth}
  \centering
  \includegraphics[width=1.1\linewidth]{fig2.png}
%   \caption{Simplified model}
\end{subfigure}%
\begin{subfigure}{.5\textwidth}
  \centering
  \includegraphics[width=1.1\linewidth]{fig3.png}
%   \caption{Simplified model}
\end{subfigure}
\end{figure}

\end{flushleft}
\end{frame}


\begin{frame}{Big-M method relaxation}
\framesubtitle{Multiple variables}
\begin{flushleft}

If you have multiple variables $\mathbf{x}_1$, ..., $\mathbf{x}_K$, and each should belong to at least one of the H-polytopes $\{ \mathbf{A}_i, \mathbf{b}_i\}$, this can also be represented using big-M method:

\begin{equation}
    \begin{cases}
    \mathbf{A}_1 \mathbf{x}_k \leq \mathbf{b}_1 + M \cdot \mathbf{1} \cdot (1 - c_{1, k}) \\
    \mathbf{A}_2 \mathbf{x}_k \leq \mathbf{b}_2 + M \cdot \mathbf{1} \cdot (1 - c_{2, k}) \\
    \mathbf{A}_3 \mathbf{x}_k \leq \mathbf{b}_3 + M \cdot \mathbf{1} \cdot (1 - c_{3, k}) \\
    c_{1, k} + c_{2, k} + c_{3, k} = 1 \\
    c_{i, k}  \in \{0, \ 1 \} \\
    k = 1,...,K
    \end{cases}
\end{equation}

Notice that the only difference from the previous example is that now we have $K$ sets of binary variables $c_{1, k}$,  $c_{2, k}$ and $c_{3, k}$.

\end{flushleft}
\end{frame}






\begin{frame}{Example: Footstep planning}
\framesubtitle{Formulation as MIQP}
\begin{flushleft}

Using big-M relaxation we can now formulate the problem as follows:

\begin{equation}
\begin{aligned}
& \underset{\mathbf{x}_1, \ ..., \ \mathbf{x}_K}{\text{minimize}}
& & ||\mathbf{x}_1 - \mathbf{x}_{\text{start}}|| + ||\mathbf{x}_K - \mathbf{x}_{\text{goal}}||, \\
& \text{subject to}
& & \begin{cases}
    \mathbf{A}_i \mathbf{x}_k \leq \mathbf{b}_i + M \cdot \mathbf{1} \cdot (1 - c_{i, k}), \ i = 1,...,N \\
    \sum_{i=1}^N c_{i, k} = 1 \\
    c  \in \{0, \ 1 \}^{N, \ K} \\
    k = 1,...,K
    \end{cases}
\end{aligned}
\end{equation}

 
\end{flushleft}
\end{frame}





\begin{frame}{Example: Footstep planning}
\framesubtitle{Evenly spaced steps}
\begin{flushleft}

In order to make the footsteps evenly spaced we add cost on the distance between consequent steps:

\begin{equation}
\begin{aligned}
& \underset{\mathbf{x}_1, \ ..., \ \mathbf{x}_K}{\text{minimize}}
& & ||\mathbf{x}_1 - \mathbf{x}_{\text{start}}|| + ||\mathbf{x}_K - \mathbf{x}_{\text{goal}}|| + w \cdot \sum_{k=1}^{K-1} ||\mathbf{x}_{i+1} - \mathbf{x}_i||, \\
& \text{subject to}
& & \begin{cases}
    \mathbf{A}_i \mathbf{x}_k \leq \mathbf{b}_i + M \cdot \mathbf{1} \cdot (1 - c_{i, k}), \ i = 1,...,N \\
    \sum_{i=1}^N c_{i, k} = 1 \\
    c  \in \{0, \ 1 \}^{N, \ K} \\
    k = 1,...,K
    \end{cases}
\end{aligned}
\end{equation}

where $w$ is a weight, a coefficient we can tune to adjust relative importance of our primary objective (starting from the point $\mathbf{x}_{\text{start}}$ and finishing at the point $\mathbf{x}_{\text{goal}}$ and our secondary objective (making the footsteps evenly spaced).  
 
\end{flushleft}
\end{frame}




\begin{frame}{Example: Footstep planning}
\framesubtitle{Code, part 1}
\begin{flushleft}

\begin{lstlisting}[language=Matlab]
n = 2;
shift_1 = 1*rand(n, 1);
shift_2 = 1*rand(n, 1);
V1 = randn(n, 6);
V2 = randn(n, 6) + shift_1;
V3 = randn(n, 6) + shift_1 + shift_2;

indices_convhull = convhull(V1');
V1 = V1(:, indices_convhull);
indices_convhull = convhull(V2');
V2 = V2(:, indices_convhull);
indices_convhull = convhull(V3');
V3 = V3(:, indices_convhull);

[A1, b1] = vert2con(V1');
[A2, b2] = vert2con(V2');
[A3, b3] = vert2con(V3');
\end{lstlisting}
 
\end{flushleft}
\end{frame}


\begin{frame}{Example: Footstep planning}
\framesubtitle{Code, part 2}
\begin{flushleft}

\begin{lstlisting}[language=Matlab]
n = 7; A = 0.35*randn(n, n);
Q = eye(n);

cvx_begin sdp
    variable P(n, n) symmetric
    minimize 0
    subject to
        P >= 0;
        A'*P*A - P + Q <= 0;
cvx_end

if strcmp(cvx_status, 'Solved')
    [abs(eig(A)), eig(A'*P*A - P), eig(P)]
else
    abs(eig(A))
end
\end{lstlisting}
 
\end{flushleft}
\end{frame}


\begin{frame}{Example: Footstep planning}
\framesubtitle{Code, part 3}
\begin{flushleft}

\begin{lstlisting}[language=Matlab]
    subject to
        for i = 1:number_of_steps
            A1*x(:, i) <= b1 + (1 - c(1, i))*bigM;
            A2*x(:, i) <= b2 + (1 - c(2, i))*bigM;
            A3*x(:, i) <= b3 + (1 - c(3, i))*bigM;
            c(1, i) + c(2, i) + c(3, i) == 1;
        end
cvx_end

plot(x(1, :)', x(2, :)', '^', 'MarkerEdgeColor', 'k', 'MarkerSize', 10, 'LineWidth', 2); hold on;
\end{lstlisting}
 
\end{flushleft}
\end{frame}




\begin{frame}{Homework}
% \framesubtitle{Parameter estimation}
\begin{flushleft}

Implement footstep planning for a biped, making sure every pair of steps lands in the same H-polytope.

\end{flushleft}
\end{frame}



\begin{frame}
\centerline{Lecture slides are available via Moodle.}
\bigskip
\centerline{You can help improve these slides at:}

\centerline{\href{https://github.com/SergeiSa/Computational-Intelligence-Slides-Fall-2020}{github.com/SergeiSa/Computational-Intelligence-Slides-Fall-2020}}


\bigskip
\centerline{Check Moodle for additional links, videos, textbook suggestions.}
\end{frame}

\end{document}
