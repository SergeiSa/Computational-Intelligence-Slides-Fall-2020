\documentclass{beamer}

 \pdfmapfile{+sansmathaccent.map}


\mode<presentation>
{
  \usetheme{Warsaw} % or try Darmstadt, Madrid, Warsaw, Rochester, CambridgeUS, ...
  \usecolortheme{crane} % or try seahorse, beaver, crane, wolverine, ...
  \usefonttheme{serif}  % or try serif, structurebold, ...
  \setbeamertemplate{navigation symbols}{}
  \setbeamertemplate{caption}[numbered]
} 

\renewcommand{\familydefault}{\rmdefault}


\hypersetup{
    colorlinks=true,
    linkcolor=blue,
    filecolor=blue,      
    urlcolor=blue,
}

\usepackage{amsmath}
\usepackage{mathtools}

\DeclareMathOperator*{\argmin}{arg\,min}

\usepackage{subcaption}

%%%%%%%%%%%%%%%%%%%%%%%%%%%%%%%%%%%%%%%%%%%%%%%%%%%



\usepackage{listings}
\usepackage{color}
\definecolor{mygreen}{rgb}{0,0.6,0}
\definecolor{mygray}{rgb}{0.5,0.5,0.5}
\definecolor{mymauve}{rgb}{0.58,0,0.82}
\lstset{ 
  backgroundcolor=\color{white},   % choose the background color; you must add \usepackage{color} or \usepackage{xcolor}; should come as last argument
  basicstyle=\footnotesize,        % the size of the fonts that are used for the code
  breakatwhitespace=false,         % sets if automatic breaks should only happen at whitespace
  breaklines=true,                 % sets automatic line breaking
  captionpos=b,                    % sets the caption-position to bottom
  commentstyle=\color{mygreen},    % comment style
  deletekeywords={...},            % if you want to delete keywords from the given language
  escapeinside={\%*}{*)},          % if you want to add LaTeX within your code
  extendedchars=true,              % lets you use non-ASCII characters; for 8-bits encodings only, does not work with UTF-8
  firstnumber=0000,                % start line enumeration with line 0000
  frame=single,	                   % adds a frame around the code
  keepspaces=true,                 % keeps spaces in text, useful for keeping indentation of code (possibly needs columns=flexible)
  keywordstyle=\color{blue},       % keyword style
  language=Octave,                 % the language of the code
  morekeywords={*,...},            % if you want to add more keywords to the set
  numbers=left,                    % where to put the line-numbers; possible values are (none, left, right)
  numbersep=5pt,                   % how far the line-numbers are from the code
  numberstyle=\tiny\color{mygray}, % the style that is used for the line-numbers
  rulecolor=\color{black},         % if not set, the frame-color may be changed on line-breaks within not-black text (e.g. comments (green here))
  showspaces=false,                % show spaces everywhere adding particular underscores; it overrides 'showstringspaces'
  showstringspaces=false,          % underline spaces within strings only
  showtabs=false,                  % show tabs within strings adding particular underscores
  stepnumber=2,                    % the step between two line-numbers. If it's 1, each line will be numbered
  stringstyle=\color{mymauve},     % string literal style
  tabsize=2,	                   % sets default tabsize to 2 spaces
  title=\lstname                   % show the filename of files included with \lstinputlisting; also try caption instead of title
}





%%%%%%%%%%%%%%%%%%%%%%%%%%%%%%%%%%%%%%%%%%%%%%%%%%%%%


\title{Linear Algebra, Four Fundamental Subspaces}
\subtitle{Computational Intelligence, Lecture 2}
\author{by Sergei Savin}
\centering
\date{Fall 2020}



\begin{document}
\maketitle


\begin{frame}{Content}

\begin{itemize}
\item Motivating questions
\item Four Fundamental Subspaces
\item Null space
\item Row space
\item Homework
\end{itemize}

\end{frame}

\begin{frame}{Motivating questions}
% \framesubtitle{Parameter estimation}
\begin{flushleft}

You have a linear operator $\mathbf A$. Try to answer the following questions:

\begin{itemize}
    \item What are all vectors this operator can produce as outputs? How to find them?
    \item Are there two inputs that make it produce the same output?
    \item Are there inputs that produce zero as an output?
    \item Are there outputs that cannot be produced? 
    \item How to find all inputs that produce a unique output? 
\end{itemize}

These questions are directly related to the idea of fundamental subspaces of a linear operator.

\end{flushleft}
\end{frame}


\begin{frame}{Four Fundamental Subspaces}
% \framesubtitle{Parameter estimation}
\begin{flushleft}

One of the key ideas in the linear algebra is that every linear operator has four fundamental subspaces:

\begin{itemize}
    \item Null space
    \item Row space
    \item Column space
    \item Left null space
\end{itemize}

\bigskip

Our goal is to understand them. The usefulness of this understating is enormous.

\end{flushleft}
\end{frame}

\begin{frame}{Null space}
% \framesubtitle{Parameter estimation}
\begin{flushleft}

Consider the following task: find all solutions to the system of equations $\mathbf{A} \mathbf{x} = \mathbf{0}$.

\bigskip

It can be re-formulated as follows: find all elements of the \emph{null space} of $\mathbf{A}$.

\begin{block}{Definition 1}
  \emph{Null space} of $\mathbf{A}$ is the set of all vectors $\mathbf{x}$ that $\mathbf{A}$ maps to $\mathbf{0}$
\end{block}

\bigskip

We will denote null space as $\mathcal{N}(\mathbf{A})$. In the literature, it is often denoted as $\text{ker}(\mathbf{A})$.

\end{flushleft}
\end{frame}


\begin{frame}{Null space}
\framesubtitle{Example}
\begin{flushleft}

Now we can find all solutions to the system of equations $\mathbf{A} \mathbf{x} = \mathbf{0}$ by using functions that generate an \emph{orthonormal basis} in the null space of $\mathbf{A}$. In MATLAB it is function \texttt{null}:

\bigskip

\texttt{N = null(A)}.

\bigskip

That is it! Space of solutions of $\mathbf{A} \mathbf{x} = \mathbf{0}$ is the span of the columns of $\mathbf{N}$, and all solutions $\mathbf{x}^*$ can be represented as $\mathbf{x}^* = \mathbf{N}\mathbf{z}$, for any $\mathbf{z}$ we get a unique solution, and for any solution a unique $\mathbf{z}$.

\end{flushleft}
\end{frame}



\begin{frame}{Row space}
\framesubtitle{Part 1}
\begin{flushleft}

Consider another task: find all solutions to the system of equations $\mathbf{A} \mathbf{x} = \mathbf{y}$.

\bigskip

Assume we have two solutions to the system: $\mathbf{x}_1$ and $\mathbf{x}_2$. We know that $\mathbf{A} \mathbf{x}_1 = \mathbf{A} \mathbf{x}_2= \mathbf{y}$, hence $\mathbf{A} (\mathbf{x}_1 - \mathbf{x}_2) = \mathbf{0}$. In other words, the difference between any two solutions lies in the null space of $\mathbf{A}$.

\bigskip

On the other hand, let $\mathbf{x}^*$ be a solution, and $\mathbf{x}^N \in \mathcal{N}(\mathbf{A})$ be a vector in the null space of $\mathbf{A}$. Then $\mathbf{x}^* + \mathbf{x}^N$ is also a solution, since $\mathbf{A} (\mathbf{x}^* + \mathbf{x}^N) = \mathbf{A} \mathbf{x}^* + \mathbf{A} \mathbf{x}^N = \mathbf{A} \mathbf{x}^* = \mathbf{y}$.

\bigskip

Therefore, the solution space is given by a single particular solution $\mathbf{x}^p \notin \mathcal{N}(\mathbf{A})$ and the whole null space of $\mathbf{A}$.

\end{flushleft}
\end{frame}


\begin{frame}{Row space}
\framesubtitle{Part 2}
\begin{flushleft}

There are infinitely many ways to chose $\mathbf{x}^p$, since if $\mathbf{x}^p \notin \mathcal{N}(\mathbf{A})$, then $(\mathbf{x}^p + \mathbf{x}^N) \notin \mathcal{N}(\mathbf{A})$, if $\mathbf{x}^N \in \mathcal{N}(\mathbf{A})$. However: 

\begin{block}{Statement 1}
we can find only a single $\mathbf{x}^p$, such that $\mathbf{x}^p \cdot \mathbf{x}^N = 0, \ \forall \mathbf{x}^N \in \mathcal{N}(\mathbf{A})$, while $\mathbf{A} (\mathbf{x}^p + \mathbf{x}^N) = \mathbf{y}$.
\end{block}

\bigskip

Such $\mathbf{x}^p$ lies in the \emph{row space} of $\mathbf{A}$. 

\begin{block}{Definition 2}
\emph{Row space} of $\mathbf{A}$ is the set of all inputs to $\mathbf{A}$ that have a zero projection onto its null space.
\end{block}

\end{flushleft}
\end{frame}





\begin{frame}{Row space}
\framesubtitle{Example}
\begin{flushleft}

Assuming $\mathbf{A} \mathbf{x} = \mathbf{y}$ has at least one solution, we can find it by solving \emph{least squares} problem, or equivalently, using a pseudo-inverse:

\bigskip

\texttt{xp = pinv(A)*y}.

\bigskip

Now all solutions to the original problem are given as $\mathbf{x}^* = \mathbf{x}^p + \mathbf{N} \mathbf{z}$, using previous notation.

\end{flushleft}
\end{frame}



\begin{frame}{Homework}
% \framesubtitle{Parameter estimation}
\begin{flushleft}

\begin{itemize}
\item Prove Statement 1.
\item Come up with a way to check if a vector is in the row space of $\mathbf{A}$.
\end{itemize}

\end{flushleft}
\end{frame}




\begin{frame}
\centerline{Lecture slides are available via Moodle.}
\bigskip
\centerline{You can help improve these slides at:}

\centerline{\href{https://github.com/SergeiSa/Computational-Intelligence-Slides-Fall-2020}{github.com/SergeiSa/Computational-Intelligence-Slides-Fall-2020}}


\bigskip
\centerline{Check Moodle for additional links, videos, textbook suggestions.}
\end{frame}

\end{document}
